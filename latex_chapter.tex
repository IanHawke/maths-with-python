\begin{chapter}{\LaTeX}

Producing high quality mathematical documents that show equations, figures and references clearly is very difficult in standard word processing packages. Instead the standard is to use \LaTeX, a ``document prepartion language''. Most notes, such as these, and mathematical papers, use \LaTeX.

\section{What is \LaTeX}

\LaTeX takes a plain text file and converts it into a document, often in PDF format, containing all the equations, figures and cross-references. In this sense it's similar to python: you start from a plain text file and \emph{build} it to produce a result.

\section{Installing or finding}

\LaTeX is free software and there's lots of available implementations. As with python, a \LaTeX document is plain text and can be written using any text editor. However, specific \LaTeX editors can help in rapidly writing documents with fewer errors.

\subsection{University machines}

Use the search function to look for \texttt{TeXnicCentre}. When first running, choose all the default options. Some of the instructions below on building a document are specific to TeXnicCentre, as it's provided on the university machines, but similar methods should work in all editors.

\subsection{Personal machines}

If using a Linux or Mac, \LaTeX should be installed automatically. There are many editors you can use -- \href{TeXMaker}{http://www.xm1math.net/texmaker/} is one example -- and you can experiment with which you prefer. If using a Windows machine, you can freely download \href{TeXnicCentre}{http://www.texniccenter.org/} which will install \LaTeX for you, or use TeXMaker as above.

\subsection{Online}

There are browser-based \LaTeX editors which can be used, such as \href{Overleaf}{https://www.overleaf.com/} and \hrefShareLaTeX}{https://www.sharelatex.com/}. There are obvious advantages -- there's no need to install anything, it's backed up externally, and it builds the document for you. The balancing disadvantages are also there -- needs a network connection, the online editor isn't as full featured as special desktop versions, and keeping your work private requires paying.
